\documentclass[11pt,]{article}
\usepackage[]{mathpazo}
\usepackage{amssymb,amsmath}
\usepackage{ifxetex,ifluatex}
\usepackage{fixltx2e} % provides \textsubscript
\ifnum 0\ifxetex 1\fi\ifluatex 1\fi=0 % if pdftex
  \usepackage[T1]{fontenc}
  \usepackage[utf8]{inputenc}
\else % if luatex or xelatex
  \ifxetex
    \usepackage{mathspec}
  \else
    \usepackage{fontspec}
  \fi
  \defaultfontfeatures{Ligatures=TeX,Scale=MatchLowercase}
\fi
% use upquote if available, for straight quotes in verbatim environments
\IfFileExists{upquote.sty}{\usepackage{upquote}}{}
% use microtype if available
\IfFileExists{microtype.sty}{%
\usepackage{microtype}
\UseMicrotypeSet[protrusion]{basicmath} % disable protrusion for tt fonts
}{}
\usepackage[margin=1in]{geometry}
\usepackage{hyperref}
\hypersetup{unicode=true,
            pdftitle={Reproducible Science with R},
            pdfauthor={Matthew K. Lau},
            pdfborder={0 0 0},
            breaklinks=true}
\urlstyle{same}  % don't use monospace font for urls
\usepackage{graphicx,grffile}
\makeatletter
\def\maxwidth{\ifdim\Gin@nat@width>\linewidth\linewidth\else\Gin@nat@width\fi}
\def\maxheight{\ifdim\Gin@nat@height>\textheight\textheight\else\Gin@nat@height\fi}
\makeatother
% Scale images if necessary, so that they will not overflow the page
% margins by default, and it is still possible to overwrite the defaults
% using explicit options in \includegraphics[width, height, ...]{}
\setkeys{Gin}{width=\maxwidth,height=\maxheight,keepaspectratio}
\IfFileExists{parskip.sty}{%
\usepackage{parskip}
}{% else
\setlength{\parindent}{0pt}
\setlength{\parskip}{6pt plus 2pt minus 1pt}
}
\setlength{\emergencystretch}{3em}  % prevent overfull lines
\providecommand{\tightlist}{%
  \setlength{\itemsep}{0pt}\setlength{\parskip}{0pt}}
\setcounter{secnumdepth}{0}
% Redefines (sub)paragraphs to behave more like sections
\ifx\paragraph\undefined\else
\let\oldparagraph\paragraph
\renewcommand{\paragraph}[1]{\oldparagraph{#1}\mbox{}}
\fi
\ifx\subparagraph\undefined\else
\let\oldsubparagraph\subparagraph
\renewcommand{\subparagraph}[1]{\oldsubparagraph{#1}\mbox{}}
\fi

%%% Use protect on footnotes to avoid problems with footnotes in titles
\let\rmarkdownfootnote\footnote%
\def\footnote{\protect\rmarkdownfootnote}

%%% Change title format to be more compact
\usepackage{titling}

% Create subtitle command for use in maketitle
\newcommand{\subtitle}[1]{
  \posttitle{
    \begin{center}\large#1\end{center}
    }
}

\setlength{\droptitle}{-2em}
  \title{Reproducible Science with R}
  \pretitle{\vspace{\droptitle}\centering\huge}
  \posttitle{\par}
  \author{Matthew K. Lau}
  \preauthor{\centering\large\emph}
  \postauthor{\par}
  \predate{\centering\large\emph}
  \postdate{\par}
  \date{June 7, 14 and 21 (Total = 4.5 hrs)}

\linespread{1.05}

\begin{document}
\maketitle

\hypertarget{why-make-your-work-reproducible-and-why-r}{%
\subsection{Why make your work reproducible and why
R?}\label{why-make-your-work-reproducible-and-why-r}}

Science is driven by the exchange of information and knowledge. A recent
study (Stodden \emph{et al.} 2018) demonstrated that only 26\% of
studies published in the journal \emph{Science} could be reproduced.
This was even more striking given that the study was conducted after
\emph{Science} had instituted its open data policy. Luckily, advances in
open-source computer languages, such as \textbf{R}, provide a way to
produce computations that can more easily document scientific research
in a transparent, easily shared way.

In this course, we will cover how to conduct \textbf{reproducible}
scientific research using the \textbf{R} programming language and
supporting software that will enable researchers to more clearly and
easily document projects. Participants will gain experience coding in
\textbf{R} using the \emph{RStudio} IDE and using other software for
reproducible research, such as the \emph{git} version control system.

\hypertarget{course-outline}{%
\subsection{Course Outline}\label{course-outline}}

The course will be a mix of demos, discussions and activities:

\textbf{Day 1}

\begin{itemize}
\tightlist
\item
  Reproducibility Framework (10 min)
\item
  \emph{RStudio} tour (10 min)
\item
  ACTIVITY: Example Project (15 min)
\item
  Getting help with R (5 min)
\item
  BREAK (10 min)
\item
  Basic plotting and function anatomy (10 min)
\item
  ACTIVITY: Write your own plot code (10 min)
\item
  Q/A and tips
\end{itemize}

\textbf{Day 2}

\begin{itemize}
\tightlist
\item
  Project Architecture (10 min)
\item
  ACTIVITY: Ecological data project (20 min)
\item
  BREAK (10 min)
\item
  Data wrangling and testing (15 min)
\item
  ACTIVITY: Test those data (20 min)
\item
  Q/A and tips
\end{itemize}

\textbf{Day 3}

\begin{itemize}
\tightlist
\item
  Style: Best Practices (5 min)
\item
  ``Backing Up'' Version Control (20 min)
\item
  ACTIVITY: Initiate git for your project (30 min)
\item
  Linear models (5 min)
\item
  ACTIVITY: Add a linear model and version your code (30 min)
\item
  BREAK (10 min)
\item
  Dependencies: R Packages, CRAN and \emph{packrat} (10 min)
\item
  ACTIVITY: Use packrat
\item
  Q/A and tips
\end{itemize}

\hypertarget{before-class-15-20-min}{%
\subsection{Before Class (15-20 min)}\label{before-class-15-20-min}}

Download and install:

\begin{enumerate}
\def\labelenumi{\arabic{enumi}.}
\tightlist
\item
  \textbf{R} \url{https://cran.r-project.org/}
\item
  \emph{RStudio}
  \url{https://www.rstudio.com/products/rstudio/download/\#download/}
\end{enumerate}

The course notes can be found in the docs directory of my course repo:
\url{https://github.com/MKLau/reprosci/archive/master.zip}

I also highly recommend the following cheat sheets for reference:
\url{https://www.rstudio.com/resources/cheatsheets/}

Also, review the code of conduct.

\hypertarget{code-of-conduct-aka.-how-to-be-a-good-community-member}{%
\subsection{Code of Conduct (aka. how to be a good community
member)}\label{code-of-conduct-aka.-how-to-be-a-good-community-member}}

Like science, open-source software development is empowered by
community. Everyone participating in this course will follow the Code of
Conduct outline by the folks at ROpenSci
(\url{https://ropensci.org/coc}).

\emph{Be considerate and respectiful of each other in speech and
actions} \emph{Contribute a safe and effective learning experience for
everyone} \emph{We all get out of this class what we put into it}

See more at \url{https://adainitiative.org}

\hypertarget{possible-advanced-topic-discussions-via-slack}{%
\subsection{Possible Advanced Topic Discussions via
Slack}\label{possible-advanced-topic-discussions-via-slack}}

\begin{itemize}
\tightlist
\item
  Creating \emph{Shiny Apps}
\item
  Writing \textbf{R} \emph{packages}
\item
  Using \emph{github}
\item
  Scientific notebooks with RMarkdown
\item
  Identifying code inefficiency with profVis
\item
  Data provenance in R
\end{itemize}


\end{document}
